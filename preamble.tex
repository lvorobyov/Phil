\documentclass[a4paper,14pt,toc=flat]{extreport} 

\usepackage{iftex,ifpdf}
\usepackage[T2A]{fontenc}
\usepackage[english,russian]{babel}
\usepackage{hyperref}
%\usepackage{mathtext}
\usepackage{newtxmath}
\usepackage[utf8x]{inputenc}
\usepackage{tempora} 
\usepackage{ucs}

\usepackage{indentfirst}
\usepackage{microtype}
\linespread{1.3}

\usepackage{geometry}
\geometry{margin=2.0cm, left=3.0cm, right=1.5cm}

\usepackage[explicit,md,tiny]{titlesec}
\titleformat{\chapter}[block]{\fillast}{\thechapter}{1em plus .5em}{\MakeUppercase{#1}}
\titlespacing{\chapter}{0pt}{-20pt}{*2.5}
\titlespacing{\section}		 {\parindent}{*4}{*1.5}
\titlespacing{\subsection}	 {\parindent}{*4}{*1.5}
\titlespacing{\subsubsection}{\parindent}{*4}{*1.5}

\newcommand{\nchapter}[1]{
	\chapter*{#1}
	\addcontentsline{toc}{chapter}{#1}
}
\providecommand*\introname{\cyr\CYRV\cyrv\cyre\cyrd\cyre\cyrn\cyri\cyre}
\providecommand*\conclusionname{%
\cyr\CYRZ\cyra\cyrk\cyrl\cyryu\cyrch\cyre\cyrn\cyri\cyre
}
\providecommand*\defsname{%
{\cyr\CYRS\cyrp\cyri\cyrs\cyro\cyrk\ %
\cyrs\cyro\cyrk\cyrr\cyra\cyrshch\cyre\cyrn\cyri\cyrishrt\ \cyri\ %
\cyru\cyrs\cyrl\cyro\cyrv\cyrn\cyrery\cyrh\ %
\cyro\cyrb\cyro\cyrz\cyrn\cyra\cyrch\cyre\cyrn\cyri\cyrishrt}
}
\newcommand{\intro}{\nchapter{\introname}}
\newcommand{\conclusion}{\nchapter{\conclusionname}}
\newcommand{\defs}{\nchapter{\defsname}}

\usepackage{enumitem}
\makeatletter
     \AddEnumerateCounter{\Asbuk}{\@Asbuk}{М)}
     \AddEnumerateCounter{\asbuk}{\@asbuk}{м)}
\makeatother
\setlist{nolistsep, noitemsep, wide, leftmargin=\parindent}
\setlist[1]{leftmargin=0pt}
\renewcommand{\labelitemi}{---}
\renewcommand{\labelitemii}{---}
\renewcommand{\labelenumi}{\asbuk{enumi})}
\renewcommand{\labelenumii}{\arabic{enumii})}
\usepackage{iitem} 

\providecommand*\appendixname{\cyr\CYRP\cyrr\cyri\cyrl\cyro\cyrzh\cyre\cyrn\cyri\cyre}
\renewcommand\appendix{%
  \setcounter{chapter}{0}
  \setcounter{section}{0}
  \renewcommand\thechapter{\Asbuk{chapter}}
  \titleformat{\chapter}[block]{\fillast}{\appendixname~\thechapter\\}{0pt}{##1}
}

\usepackage{caption} 
\captionsetup{format=plain,justification=centering,labelsep=endash}
\captionsetup[table]{justification=raggedleft,singlelinecheck=false}

% Рисунки
\usepackage{graphicx}

% Многостраничные таблицы
\usepackage{longtable,tabu,multirow}

% Альбомная ориентация страниц
\usepackage{lscape}

% Плавающие рисунки "в оборку".
\usepackage{wrapfig}

% Пакет для вывода текста в несколько колонок
\usepackage{multicol}

% Общее число страниц
\usepackage{totpages}

% Общее число рисунов и таблиц
\usepackage[figure,table]{totalcount}

% Плоское содержание
% \usepackage[toc=flat]{tocbasic}

% Пакет для листингов прогармм
\usepackage{minted}
\setminted{numbers=left, fontsize=\small, breaklines, autogobble, baselinestretch=1.1, tabsize=4, outencoding=utf8}

% Псевдокод
%\usepackage[algochapter,figure,lined]{algorithm2e}
%\SetAlgorithmName{Алгоритм}{алгоритм}{Список алгоритмов}

% Библитека для рисования фигур
\usepackage{tikz}
\usepackage{tikz-qtree}

\makeatletter
\patchcmd\tabu@startpboxmeasure
  {\aftergroup\tabu@endpboxmeasure}
  {\aftergroup\tabu@endpboxmeasure
   \color@begingroup
  }{\typeout{tabu patched}}{\typeout{tabu patch failed!}}

\patchcmd\tabu@LT@startpbox
 {\bgroup}{\bgroup\color@begingroup}
 {\typeout{tabu patched}}{\typeout{tabu patch failed!}}  
\makeatother

\makeatletter
\newcommand\tabfill[1]{%
        \dimen@\linewidth
        \advance\dimen@\@totalleftmargin
        \advance\dimen@-\dimen\@curtab
        \parbox[t]\dimen@{#1\ifhmode\strut\fi}%
}
\makeatother

\makeatletter
\def\keywords#1{\gdef\@keywords{#1}}\keywords{}
\renewenvironment{abstract}{
  \chapter*{\abstractname}
}{
  \par\vfil\null%

  \hyphenpenalty=10000

  \sloppy
  \MakeUppercase{\@keywords}%
  \tolerance=200

  \hyphenpenalty=50

  \clearpage
}
\makeatother

